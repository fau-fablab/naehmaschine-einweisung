%%%%%%%%%%%%%%%%%%%%%%%%%%%%%%%%%%%%%%%%%%%%%%%%
% COPYRIGHT: (C) 2012-2020 FAU FabLab and others
% CC-BY-SA 3.0
%%%%%%%%%%%%%%%%%%%%%%%%%%%%%%%%%%%%%%%%%%%%%%%%

\newcommand{\basedir}{./fablab-document/}
\documentclass{\basedir/fablab-document}
\renewcommand{\texteuro}{\euro}
\usepackage{ifthen}
\usepackage{xspace}
\def\tabularnewcol{&\xspace} % hässlicher Workaround von http://tex.stackexchange.com/questions/7590/how-to-programmatically-make-tabular-rows-using-whiledo


\usepackage{tabularx} % Tabelle mit teilweise gleich großen Spalten
\title{Einweisungsliste Nähmaschine Pfaff XYZ 200 \todo{genaues Modell nennen}}
\fancyfoot[C]{\hspace{7em} \small https://github.com/fau-fablab/naehmaschine-einweisung}
\fancyfoot[L]{Einweisungsliste Nr. \underline{\hspace{3em}}}

\author{Martin u.a.} % basiert auf Stickmaschinen-Einweisung von Martin

\begin{document}
%\maketitle

Ich bestätige mit meiner Unterschrift verbindlich, dass ich
\begin{itemize}
 \item die Einweisung gelesen und verstanden habe
 \item unter Anleitung erfolgreich ein Werkstück mit der Nähmaschine bearbeitet habe.
\end{itemize}

\todo{Genauer Modus der Einweisung - was darf man und ab wann soll man die Einweisung erhalten}

Die Einweisung gilt für ein Jahr und muss dann aufgefrischt werden. % (Das ist so gemäß Vorgabe Arbeitssicherheit der Uni)

\newcounter{i}
\setcounter{i}{1}

\newcommand{\leerezeile}{\hspace{2em} \tabularnewcol \hspace{3em} \tabularnewcol \hspace{2.5em} \tabularnewcol \hspace{2.5em} \tabularnewcol \vbox{\vspace{2em}} \tabularnewcol \tabularnewcol \tabularnewcol \tabularnewline \hline}

\begin{tabularx}{\textwidth}{|l|l|l|l|X|X|X|X|}
  \hline
  \textbf{Nr.} & \textbf{Datum} & \textbf{von} & \textbf{bis} & \textbf{Name} & \textbf{Unterschrift} & \textbf{Einweisender} & \textbf{Unterschrift} \\ \hline
  \whiledo{\value{i}<14}%
  {%
    \stepcounter{i} \leerezeile
  }%
  \leerezeile % doofer Workaround, eigentlich sollte das auch in der Forschleife gehen! Ohne dies wird die Spaltenbegrenzung von Spalte 1 zu weit gezeichnet.
\end{tabularx}

\end{document}
